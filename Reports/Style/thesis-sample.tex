\documentclass[12pt,twoside,openright,pdftex]{DIKU-report}

% KU-forside
\usepackage[babel, da, titelside]{ku-forside} 
\selectdanish

% Titelsideinformation
\titel{Mini-manual til stilfilen \texttt{ku-forside.sty}} %
\undertitel{Test} %
\opgave{Overspringsafhandling} % Findes kun under 'titelside'
\forfatter{Forfatter}%
\dato{Februar 2014}%
\vejleder{Vejleder} %  Findes kun under 'titelside'

\begin{document}
\maketitle % laver forsiden

\frontmatter

\tableofcontents

\mainmatter

\chapter*{Introduktion}
\addcontentsline{toc}{chapter}{Introduktion}%

Dette er ett simpelt eksempel, hvordan man bruger stilfilen
\verb|ku-forside.sty| sammen med \verb|DIKU-report.cls|.  Yderligere
oplysninger, hvordan man bruger stilfilen \verb|DIKU-report.cls|,
findes i rapporten \cite{KR14}. Denne rapport er genereret med
optioner [\verb|twoside|, \verb|openright|] hvilket betyder, at i
to-sidet udskrivning, alle kapitler starter fra h�jre side. Det vil
sige, de tomme sider i outputtet er der med vilje.

\section*{Brug af stilfilen}

Stilfilen \verb|ku-forside.sty| kan bruges til at producere en forside
til alle opgaver (f.eks.~bachelorprojekter, kandidatspecialer og
phd-afhandlinger) skrevet p� K�benhavns Universitet.  Stifilen
accepterer f�lgende optioner:
\begin{description}
\item[Sprogmuligheder:]  \texttt{da}, \texttt{en}
\item[Sprogvalg:] \verb|babel| --- unders�ger det erkl�rede sprog og s�tter pakken \verb|babel| derefter
\item[Fakultetsmuligheder:] \verb|farma|, \verb|hum|, \verb|jur|, \verb|ku|, \verb|life|, \verb|nat|, \verb|samf|, \verb|sund|, \verb|teo|
\item[Farvemuligheder:] \verb|sh|,  \verb|farve|
\item[Forsidemuligheder:] \verb|lille|, \verb|stor|, \verb|titelside|
\begin{description}
\item[\texttt{titelside}:] forsiden bliver identisk med designet p�
  \texttt{ku.dk/designmanual}
\item[\texttt{lille}:] giver et lille logo sammen med titlen p� den f�rste side
\item[\texttt{stor}:] giver et stort logo sammen med titlen p� den f�rste side
\end{description}
\end{description}

Standardindstillingerne er [\verb|da|, \verb|nat|, \verb|farve|,
  \verb|titelside|].  Disse kan �ndres, for eksempel, p� f�lgende
m�de:
\begin{quotation}
\verb|\usepackage[babel, lille, jur, sh, en]{ku-forside}|
\end{quotation}
hvilket giver et lille logo i sorthvid for juridisk fakultet og loader
\verb|babel|-pakken med engelsk som sprog.  Hvis du planl�gger at bruge denne
stilfil udenfor naturvidenskabelig fakultet eller i sorthvid, hent
arkivet \verb|ku-forside.zip| fra internettet; det indeholder de
n�dvendige billedfiler.

\begin{acknowledgements}
Den danske tekst i dette dokument er fremstillet ved hj�lp af Google
Translate.
\end{acknowledgements}

\bibliographystyle{DIKU}
\bibliography{article-sample}

\end{document}
