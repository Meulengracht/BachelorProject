\documentclass[twoside,a4paper]{report}  
\usepackage{a4}
% Not needed, since our text is english:
%\usepackage{t1enc}
\usepackage{epsf}
\usepackage{latexsym}
\usepackage{amssymb}
\usepackage[dvips]{graphicx}
\usepackage{makeidx}
\usepackage{textcomp}
\usepackage{marvosym}
\usepackage{tabularx}
\usepackage{longtable}
\usepackage{calc}

\usepackage{color}
\definecolor{linkblue}{rgb}{0, 0, 0.5}
\definecolor{urlblue}{rgb}{0, 0, 1}
\definecolor{citegreen}{rgb}{0, 0.4, 0}

\usepackage[ bookmarks=true,%
  bookmarksnumbered=true,%
  breaklinks=true,%
  colorlinks=true,%
  linkcolor=linkblue,%
  urlcolor=urlblue,%
  citecolor=citegreen]{hyperref}

\hypersetup{
  pdfauthor   = {Juha Aatrokoski, Timo Lilja, Leena Salmela, Teemu J. Takanen, Aleksi Virtanen, Philip Meulengracht},
  pdftitle    = {User-guide to KUdos System},
  pdfsubject  = {Reference Manual},
}

%\frenchspacing

% use section for all levels with autoref:
\renewcommand{\subsectionautorefname}{section}
\renewcommand{\subsubsectionautorefname}{section}


% From the LaTeX companion, PreserveBackslash:
\newcommand{\PBS}[1]{\let\temp=\\#1\let\\=\temp}
\newlength{\tablewidth}


\newenvironment{function}[3]{%
\paragraph{\texttt{#1 {\textbf{#2}} (#3)}}%
\index{#2@\texttt{#2}}%
\begin{itemize}%
}{%
\end{itemize}%
}

\newcommand{\parameter}[3]{%
\paragraph{\texttt{#1}}%
\index{#1@\texttt{#1}}%
\begin{itemize}%
\item \textbf{Purpose:} #2%
\item \textbf{Value range:} #3%
\end{itemize}%
}


\newenvironment{structdescription}{%
\begin{center}%
\begin{tabular}{p{3.5cm}|p{2.5cm}|>{\PBS\raggedright}p{\tablewidth-6\tabcolsep-6cm}}%
\textbf{Type} & \textbf{Name} & \textbf{Explanation} \\ %
}{%
\end{tabular}%
\end{center}%
}

% Arguments are caption,label
\newenvironment{longstructdescription}[2]{%
\begin{center}%
\begin{longtable}{p{3.5cm}|p{2.5cm}|>{\PBS\raggedright}p{\tablewidth-6\tabcolsep-6cm}}%
\multicolumn{3}{l}{\emph{Continued from the previous page}}\\%
\multicolumn{3}{l}{}\\%
\textbf{Type} & \textbf{Name} & \textbf{Explanation}%
\endhead%
\textbf{Type}\label{#2} & \textbf{Name} & \textbf{Explanation}\\%
\endfirsthead%
\multicolumn{3}{r}{}\\%
\multicolumn{3}{r}{\emph{Continued on the next page}}\\%
\multicolumn{3}{r}{}\\%
\caption{#1}%
\endfoot%
\multicolumn{3}{r}{}\\%
\caption{#1}%
\endlastfoot%
}{%
\end{longtable}%
\end{center}%
}

% For file formats etc. data structures. Use \formatfield for rows
\newenvironment{formatdescription}{%
\begin{center}%
\begin{tabular}{p{1.2cm}|p{2.8cm}|p{1.5cm}|>{\PBS\raggedright}p{\tablewidth-8\tabcolsep-5.5cm}}%
\textbf{Offset} & \textbf{Type} & \textbf{Name} & \textbf{Description} \\ %
}{%
\end{tabular}%
\end{center}%
}


\newcommand{\structfield}[3]{%
\hline%
\texttt{#1} & \texttt{#2} & #3 \\%
}

\newcommand{\formatfield}[4]{%
\hline%
\texttt{#1} & \texttt{#2} & \texttt{#3} & #4 \\%
}

\newcommand{\buenos}{\texttt{\textbf{BUENOS}}}
\newcommand{\kudos}{\texttt{\textbf{KUDOS}}}
\newcommand{\yams}{\texttt{\textbf{YAMS}}}

\newenvironment{filelist}[0]{%
\vspace{\baselineskip}%
\begin{center}%
\begin{tabular}{p{4cm}>{\PBS\raggedright}p{\tablewidth-4\tabcolsep-4cm}}%
\hline%
}{%
\end{tabular}%
\end{center}%
}

\newcommand{\file}[2]{\texttt{#1} \vspace{2mm} & #2 \vspace{2mm}\\}

\newcommand{\metavar}[1]{\textrm{\textsl{#1}}}

% Line break + indent for long function descriptions
\newcommand{\brtab}{\\\hspace*{1cm}}

% Optional preamble rgument, defaults to a numberless section
\newenvironment{exercises}[1][\addcontentsline{toc}{section}{Exercises}%
\section*{Exercises}\markright{EXERCISES}]{%
#1%
\begin{enumerate}%
}{%
\end{enumerate}
}

\newcounter{exercisec}[chapter]

% For normal exercises
\newcommand{\exercise}[1]{%
\item[\stepcounter{exercisec}\arabic{chapter}.\theexercisec{}.] #1%
}

% For exercises involving programming
\newcommand{\cexercise}[1]{%
\item[\stepcounter{exercisec}{\huge\Keyboard}\hspace{5mm}\textbf{\arabic{chapter}.\theexercisec{}.}] #1%
}


\newcommand{\ttilde}{\symbol{126}}

\makeindex


\newcommand{\authors}{%
Juha~Aatrokoski, Timo~Lilja, Leena~Salmela, %
Teemu~J.~Takanen, Aleksi~Virtanen and Philip Meulengracht%
}


% Page layout: move towards outer edge and add headers
%
\usepackage{fancyhdr}
\pagestyle{fancy}
\addtolength{\oddsidemargin}{0.5in}
\addtolength{\evensidemargin}{-0.5in}
% add two lines to the bottom
%\addtolength{\textheight}{\baselineskip}
% page number overhang by 1cm:
\addtolength{\headwidth}{0.5cm}
\fancyhf{}
\fancypagestyle{normal}{%
  \fancyhead[LE,RO]{\thepage}%
  \fancyhead[LO]{\slshape\rightmark}%
  \fancyhead[RE]{\slshape\leftmark}%
  \renewcommand{\headrulewidth}{0.5pt}%
}
\fancypagestyle{plain}{%
  \fancyhead{}%
  \renewcommand{\headrulewidth}{0pt}%
}


%\overfullrule=5pt
\emergencystretch=10pt

\begin{document}
\pagestyle{plain}
%\maketitle
%%%%%%%%%%%%%%%%%
% Titlepage in texinfo style
\begin{titlepage}
{
\vspace*{\stretch{1}}
\raggedright\bfseries\Huge
\texttt{KUDOS}\\
\large
\hspace{1em}is the\\
Koebenhavns Universitet Discrete Operating System\\
\rule{\textwidth}{2mm}
\raggedleft\bfseries\large
The user-guide of the \texttt{KUDOS} system\\
Version 1.0\\
\today\\
\vspace*{\stretch{2}}
\raggedright\bfseries\large
\authors{}\\
%\vspace{-2mm}
\rule{\textwidth}{1mm}
}
\newpage
\vspace*{\stretch{1}}
\noindent

\noindent\buenos{}/\kudos{} is licenced under the following "modified BSD
license" (i.e., the BSD license without the advertising clause).

\begin{flushleft}
\vspace{\baselineskip}
Copyright \copyright{} 2003--\number\year{} \authors{}
\vspace{\baselineskip}
\end{flushleft}

Redistribution and use in source and binary forms, with or without
modification, are permitted provided that the following conditions
are met:

\begin{enumerate}
\item Redistributions of source code must retain the above copyright
    notice, this list of conditions and the following disclaimer.
\item Redistributions in binary form must reproduce the above
    copyright notice, this list of conditions and the following
    disclaimer in the documentation and/or other materials provided
    with the distribution.
\item The name of the author may not be used to endorse or promote
    products derived from this software without specific prior
    written permission.
\end{enumerate}

THIS SOFTWARE IS PROVIDED BY THE AUTHOR ``AS IS'' AND ANY EXPRESS
OR IMPLIED WARRANTIES, INCLUDING, BUT NOT LIMITED TO, THE IMPLIED
WARRANTIES OF MERCHANTABILITY AND FITNESS FOR A PARTICULAR PURPOSE
ARE DISCLAIMED. IN NO EVENT SHALL THE AUTHOR BE LIABLE FOR ANY
DIRECT, INDIRECT, INCIDENTAL, SPECIAL, EXEMPLARY, OR CONSEQUENTIAL
DAMAGES (INCLUDING, BUT NOT LIMITED TO, PROCUREMENT OF SUBSTITUTE
GOODS OR SERVICES; LOSS OF USE, DATA, OR PROFITS; OR BUSINESS
INTERRUPTION) HOWEVER CAUSED AND ON ANY THEORY OF LIABILITY,
WHETHER IN CONTRACT, STRICT LIABILITY, OR TORT (INCLUDING
NEGLIGENCE OR OTHERWISE) ARISING IN ANY WAY OUT OF THE USE OF THIS
SOFTWARE, EVEN IF ADVISED OF THE POSSIBILITY OF SUCH DAMAGE.

\end{titlepage}
%%%%%%%%%%%%%%%%%


\tableofcontents

\cleardoublepage
\pagestyle{normal}
\pagenumbering{arabic}
\setlength{\tablewidth}{\linewidth-1cm}

\chapter{Introduction}

\kudos{} is a derivative project from \buenos{} which aside from supporting
the old \buenos{} system, now also supports the Intel x86-64 Architecture.
The new operating system, \kudos{}, is meant also meant as a exercise base
for the operating system course, but with a more common platform in mind.
Unlike its old project \buenos{}, \kudos{} will run on more recent real-world
hardware\footnote{In theory.} without any special architecture, but rather 
on your own laptop at home.

The system is kept structurally intact to \buenos{}, however all platform-specific
code has been split up into sub-directories, and is controlled with the makefile.
\kudos{} systems are ready for multi-core, however support for starting other
application processors has not been implemented, while the \buenos{} part
has the multi-core support. Both \buenos{} and \kudos{} also provides skeleton code 
for threading, wide variety of synchronization primitives, userland support and
proccesses.A simple custom filesystem and code for networking is also 
provided (However no network-card drivers are provided). 

To keep both 32 bit code (the mips project) and 64 bit code (the x86-64 project)
under the same roof, many modifications had to be made to the old \buenos{} 
project, and thus the structure of the project has changed partially. It has also
changed the procedure for starting up the new operating system, \kudos{}.

Just like \buenos{}, the main idea of the system is to give you a real, working
multiprocessor operating system kernel which is as small and simple as
possible. To boot \kudos{} on a real computer all you would need is simply
GRUB2 as a bootloader, and then make sure to add \kudos{} as a boot
entry into GRUB2. \kudos{} now supports your everyday intel 64 bit
architecture and thus can run on your everyday computer. No code
modifcations would be neccessary. 

\section {Tools Needed}
If you are a student participating on an operating systems project
course, the course staff has probably already set up a development
environment for you. If they have not, look below to get the appropriate
tools needed.

\subsection{Tools needed for the MIPS Kernel}

To run it on a virtual machine:

\begin{enumerate}
\item A virtual machine (VirtualBox, Bochs or QEmu) with linux
\item The \kudos{} source code
\item A Mips-32 cross compiler
\item \yams{} (see below for details)
\end{enumerate}

\subsection{Tools needed for the x86-64 Kernel}
To run the x86-64 kernel you have two options, you can either run
it on a virtual machine (you should get a premade distrubuted disk image
which contains both a kudos boot and a linux boot from your teacher),
or you can run it directly on your own computer

To run it on a virtual machine:

\begin{enumerate}
\item A virtual machine (VirtualBox, Bochs or QEmu)
\item A linux distribution with GRUB2 on the virtual machine
\item The \kudos{} source code
\item A x86-64 cross compiler
\end{enumerate}

To run it on your own computer:

\begin{enumerate}
\item The \kudos{} source code
\item A x86-64 cross compiler
\item A computer with GRUB2
\item A small raw partition that can
be used for tfs filesystem.
\item You need knowledge of how
to add new boot-entries into GRUB,
or learn so from google.
\end{enumerate}

\section{Expected Background Knowledge}

Since the \kudos{} system is written using the C programming
language, you should be able to program in C. For an introduction on C
programming, see the classical reference \cite{kandr}. You also need
to know quite much about programming in general, particularly about
procedural programming.

We also expect that you have taken a lecture course on operating
systems or otherwise know the basics about operating systems. You can
still find OS textbooks very handy when doing the exercises. We
recommend that you acquire a book by Stallings \cite{stallings} or
Tanenbaum \cite{tanenbaum}.

Since you are going to interact directly with the hardware quite a
lot, you should know something about hardware. A good introduction on
this can be found in the book \cite{patterson}.

Since kernel programming generally involves a lot of synchronization
issues a course on concurrent programming is recommended. One good
book on this field is the book by Andrews \cite{andrews}. These issues
are also handled in the operating systems books by Stallings and
Tanenbaum, but the approach is different.

\section{\kudos{} for teachers}

As stated above, the \kudos{} system is meant as an assignment
backbone for operating systems project courses. This document, and
the roadmap, while primarily acting as reference guide to the system, 
is also designed to support project courses. The document is ordered 
so that various kernel programming issues are introduced in sensible 
order and exercises (see also \autoref{sec:exercises}) are provided for
each subject area.

While the system as such can be used as a base for a large variety of
assignments, this document works best if assignments are
divided into five different parts as follows:

\begin{enumerate}

\item \textbf{Synchronization and Multiprogramming.} Various
multiprogramming issues relevant on both multiprocessor and
uniprocessor machines are covered in the threading chatper and
the synchroniation chapter.

\item \textbf{Userland}. Userland processes, interactions between
kernel and userland as well as system calls are covered in the 
userland chapter.

\item \textbf{Virtual Memory}. The current virtual memory support
mechanisms in \kudos{} are explained in the virtual memory chapter, 
which also gives exercises on the subject area.

\item \textbf{Filesystem}. Filesystem issues are covered in the 
filesystems chapter.

\item \textbf{Networking}. Networking in \kudos{} is explained in the
networking chapter, but note that the base system doesn't include a
driver for the network interface. Thus it is recommended to provide
one as a binary module for students or use the device interface as
a part of this round and let students implement one.

\end{enumerate}

\subsection{Preparing for \kudos{} Course}

To implement an operating systems project course with \kudos{}, the course
administrators need to decide whether the mips or the x86-64 architecture will
be used for testing, or both architectures will be tested. As a minimum the below
items should be provided:

\begin{itemize}

\item Provide students with a development environment with precompiled
\yams{}, MIPS32 ELF cross compiler and x86-64 cross compiler and GRUB2 as bootloader. 
See \yams{} usage guide for instructions on setup of \yams{} and 
the cross compiler environment.

\item Decide which exercises are used on the course, how many points
they are worth and what are the deadlines.

\item Decide any other practical issues (are design reviews compulsory
for students, how many students there are per group, etc.)

\item Familiarize the staff with \kudos{}, GRUB2 and \yams{}.

\item Introduce \kudos{} to the students.

\end{itemize}

\section{Exercises}
\label{sec:exercises}

Each chapter in the roadmap contains a set of exercises. Some of
these are meant as simple thought challenges and some as much more
demanding and larger programming exercises.

The thought exercises are meant for self study and they can be used to
check that the contents of the chapter were understood. The
programming exercises are meant to be possible assignments on
operating system project courses.

The exercises look like this:

\begin{exercises}[\vspace{\baselineskip}]

\exercise{This is a self study exercise.}

\cexercise{This is a programming assignment. They are indicated with a
bold exercise number and a keyboard symbol.}

\end{exercises}

\section{Contact Information}

Latest versions of \kudos{} can be found at the project homepage:

%\begin{verbatim}
\vspace{\baselineskip}
\url{http://www.ku.dk/projects/kudos/}
\vspace{\baselineskip}
%\end{verbatim}

Latest versions of \yams{} for the mips-kernel can be downloaded from the
project home-page at:

%\begin{verbatim}
\vspace{\baselineskip}
\url{http://www.niksula.hut.fi/u/buenos/}
\vspace{\baselineskip}
%\end{verbatim}

Authors from \kudos{} can be contacted on mail lpz849@alumni.ku.dk.
Currently there is no publicly available mailing list to subscribe,
but one may be created if needed.

Authors from the old project (\buenos{}) can be contacted through the
mailing list \texttt{buenos@cs.hut.fi}, if you have questions about the old
project.

\chapter{Using KUdos}
\label{sec:usage}

\section{Installation and Requirements}
\label{sec:install}
\index{GRUB}
\index{YAMS}
\index{Make}
\index{GNU Make}

\underline{The \kudos{}-mips system requires the following software to run:}
\begin{itemize}
\item \yams{} machine simulator, version 1.3.0 or above\footnote{A
previous version of \yams{} can also be used if the output format is
set to ``binary'' in the linker script \texttt{ld.script}}
\item GNU Binutils for \texttt{mips-elf} target
\item GNU GCC cross-compiler for \texttt{mips-elf} target
\item GNU Make
\end{itemize}

First you have to set up the \yams{} machine simulator. From \yams{}
documentation you can find instructions on how to set up Binutils and
GCC cross-compiler.

After the required software is installed installing \kudos{} is
straightforward: you simply extract the \kudos{} distribution tar-file
to some directory.
\newline 
\newline
\underline{The \kudos{}-x86-64 system requires the following software to run:}
\begin{itemize}
\item GNU Binutils for \texttt{x86\_64-elf} target
\item GNU GCC cross-compiler for \texttt{x86\_64-elf} target
\item GNU Make
\item GNU GRUB or GRUB2
\end{itemize}

First you have to set up a virtual machine with a linux destribution. Then you should
acquire the source code of binutils and gcc, and cross-compile it for x86\_64 as a target.
For extra help it is possible to consult this webpage to help with the build-process of GCC

%\begin{verbatim}
\vspace{\baselineskip}
\url{http://wiki.osdev.org/GCC_Cross-Compiler}
\vspace{\baselineskip}
%\end{verbatim}

After the required software is installed installing \kudos{} is
straightforward: you simply extract the \kudos{} distribution tar-file
to some directory and build it.

\section{Compilation}
\label{sec:compiling}
\index{compiling!the system}

To compile \kudos{} for MIPS, you want to invoke the command 
\texttt{make mips} in the main directory of \kudos{}

To compile \kudos{} for x86-64, you want to invoke the command 
\texttt{make x64} in the main directory of \kudos{}

You can compile both MIPS and x86-64 systems  by invoking \texttt{make} in the
main directory of the \buenos{}. After compiling the system, you should have a binary named
\texttt{kudos\_64} or \texttt{kudos\_mips} in the main directory.

\section{Booting the System}
\label{sec:booting}
\index{invoking YAMS@invoking \texttt{YAMS}}
\index{YAMS!invoking}

\subsection{Booting the mips system}
After the system has been properly built, you can start \yams{} with
\kudos{} binary by invoking
\begin{verbatim}
yams kudos_mips
\end{verbatim}
at the command prompt. If you want to give boot arguments to the
system, see the section about boot arguments in the roadmap.

\index{yamst@\texttt{yamst}}
If you are using the default \yams{} configuration that is shipped
with \kudos{}, you have to start the \texttt{yamst} terminal tool
before invoking \texttt{yams}. The terminal tool provides the other
end-point of the \texttt{yams} terminal simulation. To start
\texttt{yamst}:
\begin{verbatim}
yamst -lu tty0.socket
\end{verbatim}
in another terminal (e.g. in another XTerm window).

\subsection{Booting the x86-64 system}
After the system has been properly built, you need to update the
onboard installation of \kudos{}, which is simply a fat32 partition 
on the virtual machine that contains the \kudos{} kernel in the path 
/boot/kUdOS. A make-command has been provided to make
this process easier, you can update the \kudos{} installation by simply
invoking \texttt{make install} in the root directory of \kudos{} after a 
successful build.

The command simply mounts the fat32 partition, and copies over the local
image file (kudos\_64 kernel image) to the fat32 partition and overwrites the 
old image.

Then to run the new kernel build, you must restart your virtual machine
and choose "kUdOS Live Build" in the GRUB2 boot menu. If you wish to
pass boot arguments to \kudos{}, you may want to only use the arrow buttons
to select \kudos{}, and then press 'e'.


\section{Compiling Userland Programs}
\label{sec:compilinguserland}
\index{compiling!userland programs}
\index{userland!compiling programs}

\subsection{Compiling}
Userland programs are compiled using the same cross-compiler that is
used for compiling \kudos{}. To compile userland binaries go to
the userland directory \texttt{tests/} and invoke \texttt{make}.
Every userland-program is compiled into two different targets, one
for mips (ex. halt), and one for x86-64 (ex. halt64). All userland programs
for the x86-64 architecture gets the extension '64'.

\subsection{Installing - mips}
To run compiled programs they need to be
copied to a \yams{} disk, where \kudos{} can find
them. TFS-filesystem (see filesystem chapter) is implemented and a tool
(see the TFS-tool chapter) is provided to copy binaries from host
filesystem to \kudos{} filesystem. 

\subsection{Installing - x86-64}
To run compiled programs they need to be
copied to a image created by the tfs-tool in the \texttt{utils/} directory.
Afterwards, this image now needs to be written to disk, so \kudos{} can
find the find this particular tfs image.

A makefile command has been provided to simplify this procedure, you can
copy the image file created by \texttt{tfstool} in the \texttt{utils/} directory
to the \kudos{} main directory, and then invoke the command \texttt{make copy IMG=<image>}
This will install the image file you provide onto the pre-defined empty partition on disk using the
dd linux command (disk write/read utility). It will override an old installation of the same image, so
remember that if you had previously copied a file, it wont exist after you have written a new image
using the \texttt{make copy <image>}

\section{Using the Makefiles}
\label{sec:makefiles}
\index{makefiles}

\kudos{} has two makefiles that are used to build the binaries needed
by \kudos{}. The system makefile builds the \kudos{} binaries and the
submission archive needed to submit the exercises for reviewing. This
makefile is in the \kudos{} main directory and is called
\texttt{Makefile}. The other makefile is the makefile responsible for
building the userland binaries. This makefile is in the
\texttt{tests/} directory.

\subsection{System Makefile}

\kudos{} uses somewhat unorthodox monolithic makefile. The system
is based on Peter Miller's paper~\cite{miller}. \kudos{} is divided
to modules that correspond to the directory structure of the source
code tree (see kernel overview chapter).

\index{FILES@\texttt{FILES}}

The files in the module directories are built to \kudos{} binaries.
These module directories have a file called \textit{module.mk} that
contains the name of the module and list of the files included from
this module. So, for example, the \texttt{module.mk} in the lib
directory:
\begin{verbatim}
# Makefile for the lib module

# Set the module name
MODULE := lib

FILES := libc.c xprintf.c rand.S bitmap.c debug.c

SRC += $(patsubst %, $(MODULE)/%, $(FILES))
\end{verbatim}

If you add files to your system, you have to modify only the FILES
variable. There should be no need to change anything else.

The \kudos{} makefile system also contains some platform-specific modules now aswell,
which means if you wanted to extend one of the platform-specific builds, we can take a
look at a platform specific module. For example, the \texttt{module.mk} in the init directory:
\begin{verbatim}
# Makefile for the init module

# Set the module name
MODULE := init/x86\_64

FILES := _boot.S main.c

X64SRC += $(patsubst %, $(MODULE)/%, $(FILES))
\end{verbatim}

Here it is easy to see, that instead of adding files to the "SRC" variable, we add the specifically 
to the "X64SRC", which means they are only compiled for the x86-64 target.


\index{MODULES@\texttt{MODULES}}
\index{CHANGEDFLAGS@\texttt{CHANGEDFLAGS}}
\index{IGNOREDREGEX@\texttt{IGNOREDREGEX}}

The main makefile is in the main directory and named
\texttt{Makefile}. There are few features in the Makefile that you
have to be aware of. In the unlikely event that you wish to add a new
module (directory) to the system, you have to modify the
\texttt{MODULES} variable by extending it with the module name.
Remember that this name must be same as the directory where the module
is. When you do your exercises, you have to wrap them with
CHANGED\_$n$ C-Preprocessor variables. You can define these variables
by modifying the \texttt{CHANGEDFLAGS} variable. The variable
\texttt{IGNOREDREGEX} is used when you build your submission archive
on returning your assignment. The variable contains a regular
expression pattern with which the matching files are filtered out from
the actual submission archive.

The following targets can be built by the system makefile:

\nopagebreak
\begin{description}
\item[\texttt{all}]~

The default, builds both the \kudos{}-mips binary, the
\kudos{}-x86\_64 binary and the \texttt{tfstool}.

\item[\texttt{mips}]~

Builds both the \kudos{}-mips binary and the \texttt{tfstool}.

\item[\texttt{x64}]~

Builds both the \kudos{}-x86\_64 binary and the \texttt{tfstool}.

\item[\texttt{install}]~

Updates the onboard installation of the \kudos{}-x86\_64 binary.
The actual steps performed is:
mount fat32 partition (pre-setup)
copy local kudos64-binary to fat32 partition, overwrite old
umount fat32 partition

\item[\texttt{copy IMG=<image>}]~

Writes the given image file to the onboard raw partition (also pre-setup),
so that \kudos{} can locate this partition and use it as filesystem.
The actual steps performed is:
dd if=IMG of=RAWPARTITION
It just writes the image-file to the start of the raw partition using DD.

\item[\texttt{util/tfstool}]~

Build the tfstool utility.

\item[\texttt{clean}]~

Clean the compilation files.

\item[\texttt{real-clean}]~

Clean also the depedency files.

\item[\texttt{submit-archive PHASE=$n$}]~

Builds \texttt{submit-$n$.tar.gz} in the parent directory of the main
buenos directory.  The variable $n$ indicates the submission round
number (default is 1).

\end{description}

\subsection{Userland Makefile}

\index{userland!compiling programs}
\index{compiling!userland programs}
\index{SOURCES@\texttt{SOURCES}}

To build userland binaries go to the \texttt{tests/} subdirectory and
invoke \texttt{make}. There are no special targets and the makefile is
organised so that every binary is built. If you wish to add your own
binaries to the makefile, add your source files to the
\texttt{SOURCES} variable at the beginning of the makefile.


\section{Using Trivial Filesystem}
\label{sec:tfstool}

\index{creating!a TFS volume}
\index{TFS!creating a volume}

For easy testing of \kudos{}, some method is needed to transfer files
to the filesystem in \kudos{}. The Unix based utility program,
\texttt{tfstool}, which is shipped with \kudos{}, achieves this goal.
\texttt{tfstool} can be used to create a Trivial Filesystem (TFS, see
\autoref{sec:tfs} for more information about TFS) to a given file, to
examine the contents of a file containing TFS and to transfer files to
the TFS. \kudos{} implementation of TFS does not include a way to
initialize the filesystem so using \texttt{tfstool} is the only way to
create a new TFS. \texttt{tfstool} is also used to move userland
binaries to TFS. When you write your own filesystem to \kudos{}, you
might find it helpful to leave TFS intact. This way you can still use
\texttt{tfstool} to transfer files to the \kudos{} system without
writing another utility for your own filesystem.

The implementation of the \texttt{tfstool} is provided in the
\texttt{util/} directory. The \kudos{} makefile can be used to
compile it. Note that \texttt{tfstool} is compiled with the native compiler,
not the cross-compiler used to compile \kudos{}. The implementation
takes care of byte-order conversion if needed.

To get a summary of the arguments that \texttt{tfstool} accepts you
may run it without arguments. The accepted commands are also explained
below:

\begin{description}

\item[\texttt{create \metavar{filename} \metavar{size}
\metavar{volume-name}}]~

Create a new TFS to file \metavar{filename}. The total size of the
file system will be \metavar{size} 512-byte blocks. Note that the
three first blocks are needed for the TFS header, the master directory
and the block allocation table and therefore the minimum size for the
disk is 3. The created volume will have the name
\metavar{volume-name}.

Note that the number of blocks must be the same as the setting in
\texttt{yams.conf}

\item[\texttt{list \metavar{filename}}]~

List the files found in the TFS residing in \metavar{filename}.

\item[\texttt{write \metavar{filename} \metavar{local-filename}
[\metavar{TFS-filename}]}]~

Write a file from the local system (\metavar{local-filename}) to the
TFS residing in the file \metavar{filename}. The optional fourth
argument specifies the filename in TFS. If not given,
\metavar{local-filename} will be used.

Note that you probably want to give a TFS filename, since otherwise
you end up with a TFS volume with files named like
\texttt{tests/foobar}, which can cause confusion since TFS does not
support directories.

\item[\texttt{read \metavar{filename} \metavar{TFS-filename}
[\metavar{local-filename}]}]~

Read a file (\metavar{TFS-filename}) from TFS residing in the file
\metavar{filename} to the local system. The optional fourth argument
specifies the filename in the local system. If not given, the
\metavar{TFS-filename} will be used.

\item[\texttt{delete \metavar{filename} \metavar{TFS-filename}}]~

Delete the file with name \metavar{TFS-filename} from the TFS residing
in the file \metavar{filename}.

\end{description}


\section{Starting Processes}

To start a userland process in \kudos{} you have to
\begin{enumerate}
\item have kudos{} kernel binary (compile if it doesn't already
  exist).
\item have the userland binary (compile if it doesn't exist).
\item have a filesystem disk image (use \texttt{tfstool} to create
  this).
\item copy the userland binary with \texttt{tfstool} to the file
  system image.
\item write the file system image using \texttt{make copy IMG=image} if needed
\item boot the system with proper boot parameters (see boot arguments section).
\end{enumerate}

\kudos{} is shipped with simple userland binary \texttt{halt} which
invokes the only already implemented system call \texttt{halt}. Here
is an example of how to compile \kudos{}-mips, install the userland binary
and boot the system.

\begin{verbatim}
  cd kudos
  make mips
  make -C tests/
  util/tfstool create store.file 2048 disk1
  util/tfstool write store.file tests/halt halt
  yamst -lu tty0.socket # (in another window, socket is in the main dir)
  yams kudos-mips 'initprog=[disk1]halt'
\end{verbatim}

After running the above commands the \kudos{} output should go to the
window where you started \texttt{yamst}. The \texttt{halt} program
merely shutdowns the system, thus \yams{} should exit with the message
\texttt{"YAMS running...Shutting down by software request"}.

\clearpage
\phantomsection
\addcontentsline{toc}{chapter}{\bibname}
\begin{thebibliography}{Tanenbaum}

\bibitem[Andrews]{andrews} 
Andrews, G. R., \emph{Foundations of multithreaded, parallel and
distributed programming}, ISBN 0-201-35752-6, Addison-Wesley Longman,
2000

\bibitem[Patterson]{patterson} 
Patterson, D. A., \emph{Computer organization and design: the
hardware/software interface}, ISBN 1-55860-491-X, Morgan Kaufmann
Publishers, 1998

\bibitem[Stallings]{stallings}
Stallings, W., \emph{Operating Systems: Internals and Design
Principles}, 4th edition, ISBN 0-13-032986-X, Prentice-Hall, 2001

\bibitem[K\&R]{kandr}
Kernighan B. W., Ritchie D. M., \emph{The C Programming Language}, 2nd
Edition, ISBN \mbox{0-13-110362-8}, Prentice-Hall, 1988

\bibitem[Tanenbaum]{tanenbaum}
Tanenbaum, A. S., \emph{Modern Operating Systems}, 2nd edition, ISBN
\mbox{0-13-031358-0}, Prentice-Hall, 2001

\bibitem[Miller]{miller}
Miller, Peter, \emph{Recursive Make Considered Harmful},
\url{http://www.tip.net.au/~Emillerp/rmch/recu-make-cons-harm.html}


\end{thebibliography}

\printindex

\end{document}
